\documentclass{article}
\usepackage{amsmath}
\usepackage{amsthm}
\usepackage{amsfonts}
\usepackage{amssymb}
\usepackage{mathtools}
\title{ELMO 2019 P5}
\date{\today}

\newtheorem*{problem}{Problem}
\newtheorem*{solution}{Solution}
\newtheorem*{exploration}{Exploration}
\newtheorem*{tags}{Tags}

\begin{document}
	
	\maketitle
	
	\begin{problem}
		Let $\mathbb{S}$ be a nonempty set of positive integers such
		that, for any (not necessarily distinct) integers a and b in $\mathbb{S}$, the number ab + 1 is also in $\mathbb{S}$.
		Show that the set of primes that do not divide any element of $\mathbb{S}$ is finite.
	\end{problem}
	
	\begin{solution}
		
		Let p be a prime which doesnt divide any element in $\mathbb{S}$ but has at least 2 different residues in $\mathbb{S}$. Let $Q$ be the set of all residues of numbers in S modulo p. Then we have $1<|Q|<p$. Note that  if $p<7$ this can never hold. We will consider all primes $\geq 7$. Note that if $1 \in Q$ then for all $a \in Q$, we have $a+1 \in Q$. This is a contradiction as we will have every number modulo p by repeating this process. Hence $1 \not \in Q$ .
		
		Now note that if $a \in Q$ then  $a \cdot Q +1 = Q$ as $a \cdot Q +1$ has same cardinality as of Q and every element of $a \cdot Q +1$ is an element of Q. Hence for all $a,b \in Q$, we have 
		$a \cdot Q = Q-1 = b \cdot Q$.
		
		Now note that $(ab)Q = a(bQ) = a(Q-1) = aQ- a = Q-a-1$, similarly $(ab)Q = b(aQ) = b(Q-1) = bQ- b = Q-b-1$ hence $Q-a = Q-b $, therefore  $\forall c \in Q , c+a-b \in Q$. If we choose distinct a and b we get $c+t(a-b) \in Q$ for all $t$, note that this implies all residues  are in Q which is a contradition.
		Hence if a prime has at least 2 residues in $\mathbb{S}$, then it has all the residues. Hence all primes which are greater than second smallest element has an element in $\mathbb{S}$ that it divides. QED

	\end{solution}
	
	\begin{exploration}
		 let g be a primitive root of prime p, index set Q by powers of g. We are only talking in field $\mathbb{F}_p$ from now on.So
		$Q= \{g^{a_i}: 0 \leq i < k \}$. Let $d= min(\{a_{i}-a_{i-1}\} : 0<i<k)$. Then note that if $x \in Q$ then so is $x \cdot g^d$. Now note that for all $0 \leq i < k $ we have $g^{a_i+d} \in Q$, this implies $a_{i+1}-a_{i}=d$ for all possible i and $a_{k-1}+d=a_0+(p-1)$ so $k \cdot d = p-1$. Hence set Q is $g^{a_0} \cdot \{1,g^d,g^{2d}, \dots g^{(k-1)d}\}$. Now as $g^{2a_0}+1 \in \mathbb{Q}$, we get $g^{2a_0}+1 \equiv g^{a_0 + cd} \mod p$ for some $0 \leq c< k$. Post that it's somehow easy to solve you show that -1 is a power of g , so we have $(1+1/b) \in S$ , so $b+2 \in S$, this solves the problem
		Basically the solution , but I used the $a^2+1$ and then $ab+1$ identity repeatedly to rederive many of these identities. The relation is too strong tbh and a few different integers should explore the whole modulo (0 ,1 implies whole modulo is reachable). Now it is not always reachable because the most basic case where $a^2 +1 \equiv a \mod p$ then we can choose all numbers $a \mod p$ and be done. But 2 seemed sufficient to get all modulos (verified lazily with computer till like primes$<$1000). One can write this solution w/o primitive roots but they help guide the solution and show dark in light if you dont see the trick.
	\end{exploration}
	
	\begin{tags}
		NT ,  Algebra , ab+1
	\end{tags}
	
\end{document}
