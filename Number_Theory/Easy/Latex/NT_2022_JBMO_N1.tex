\documentclass{article}
\usepackage{amsmath}
\usepackage{amsthm}
\usepackage{amsfonts}
\title{NT 2022 JBMO N1}
\date{\today}

\newtheorem*{problem}{Problem}
\newtheorem*{solution}{Solution}
\newtheorem*{exploration}{Exploration}
\newtheorem*{tags}{Tags}

\begin{document}
	
	\maketitle
	
	\begin{problem}
		Determine all pairs $(k,n)$ of positive integers that satisfy
		\[1! + 2! + \dots + k! = 1 + 2 + \dots + n \]
	\end{problem}
	
	\begin{solution}
		Note that of $k<7$ only 3 solutions exist namely $(1,1)$ , $(2,2)$ and $(5,17)$ . If $k \geq 7 $ then in this case we have 
		\[2*(1! + 2! + \dots + k!) = n(n+1)\]
		Taking modulo $7$ both sides we get
		$ 3 \equiv n(n+1) \mod 7$ , this is a contraditction as ($n(n+1)) $ can never have a residue $3$ mod $7$ . Hence these are the only solutions
	\end{solution}
	
	\begin{exploration}
		This equation appears to be solvable by modular arithmetic. Just a matter of choosing right mod, tried with a computer and 7 works just fine. Another intuition on why modular arithmetic should work is if $k \geq p$ for some $p$ then left side is always constant modulo $p$ while left side is 
		\[\frac{1}{2}\left( \left( n + \frac{1}{2} \right)^{2} - \dfrac{1}{4}\right)   \]
		This should only take half the values as quadratic residues can only take half the values modulo $p$. Since left and right side feels independent, each value of $p$ has a $50 \%$ chance of working. Hence a modular search for contradiction feels natural.
	\end{exploration}
	
	\begin{tags}
		Number Theory ,  Diophantine equation , Factorial , JBMO , Shortlist , Triangular Numbers
	\end{tags}
	
\end{document}