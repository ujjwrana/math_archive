\documentclass{article}
\usepackage{amsmath}
\usepackage{amsthm}
\usepackage{amsfonts}
\usepackage{amssymb}
\title{NT Berman Unity}
\date{\today}

\newtheorem*{problem}{Problem}
\newtheorem*{solution}{Solution}
\newtheorem*{exploration}{Exploration}
\newtheorem*{tags}{Tags}

\begin{document}
	
	\maketitle
	
	\begin{problem}
		Let $p$ be an odd prime and $x$ be an integer such that $p \mid x^3 - 1)$ but $p \nmid x-1$. Prove that
		\[
		p \mid (p-1)! \left( x - \frac{x^2}{2} + \frac{x^3}{3} - \dots - \frac{x^{p-1}}{p-1} \right) 
		\]
		\textit{John Berman}
	\end{problem}
	
	\begin{solution}
		First of all we note that since $ \mathbb{Z}/p\mathbb{Z}$ is a field therefore we can deal with rational numbers just fine. Hence our initial condition is equivalent to 
		\[
		\left( x - \frac{x^2}{2} + \frac{x^3}{3} - \dots - \frac{x^{p-1}}{p-1} \right)  \equiv 0 \mod p
		\]
		which is equivalent to 
		\[
		\sum_{i=1}^{p-1} \dfrac{(-1)^i x^ i }{i} \equiv 0 \mod p
		\]
		Let $P(x)$ be the polynomial $\sum_{i=1}^{p-1} \dfrac{(-1)^i x^ i }{i}$.
		Since $p|(x^3 -1 )$ but $p\nmid x-1 $ this implies $p|(x^2+x+1)$ . Now note that $x^{3n+c} \equiv  x^c  \mod p $.
				Note that order of x with respect to p is 3, hence $3|p-1$. As $p$ is an odd prime,we get $6|p-1$. Hence $p=6m+1$ for some natural $m$. 
		 Hence 
		\[
		P(x) \equiv \left( \left(  \sum_{i=0}^{i=2m-1} \dfrac{(-1)^{3i+3} }{3i+3}  \right)  + \left(  \sum_{i=0}^{i=2m-1} \dfrac{(-1)^{3i+1} }{3i+1} \right) x + \left(  \sum_{i=0}^{i=2m-1} \dfrac{(-1)^{3i-2} }{3i+2} \right) x^2   \right)  \mod p
		\]
		Let $A=\left(  \sum_{i=0}^{i=2m-1} \dfrac{(-1)^{3i+3} }{3i+3}  \right) , B = \left(  \sum_{i=0}^{i=2m-1} \dfrac{(-1)^{3i+1} }{3i+1} \right)   , C =  \left(  \sum_{i=0}^{i=2m-1} \dfrac{(-1)^{3i-2} }{3i+2} \right) $ . 
		
		We will prove that $A \equiv B \equiv C \mod p$. which will prove $P(x) \equiv 0 \mod p$  which will prove our original proposition . We will divide our proof into 2 parts , first we prove $A \equiv B \mod p$ , in second part we will prove $A \equiv C \mod p$. 

		Proof that $A \equiv B \mod p$ $\to$
		
		Note that  as $1/a \equiv -1/(p-a) \mod p$, we get
		\[
		\left(  \sum_{i=0}^{i=2m-1} \dfrac{(-1)^{3i+3} }{(3i+3)}  \right) \equiv \left(  \sum_{i=0}^{i=2m-1} \dfrac{(-1)^{3i+2} }{6m+1 -(3i+3)}  \right) \mod p
		\]
		which implies
		\[
		A \equiv \left(  \sum_{i=0}^{i=2m-1} \dfrac{(-1)^{6m-2-3i} }{6m-2-3i}  \right) \mod p
		\]
		as we know $\sum_{i=a}^{i=b} f(i) = \sum_{i=a}^{i=b} f(a+b-i)$, we get
		\[
		A \equiv \left(  \sum_{i=0}^{i=2m-1} \dfrac{(-1)^{3i+1} }{3i+1}  \right) \equiv B \mod p
		\]
		Proof that $A \equiv C \mod p$ $\to$
		
		We will prove that $3A \equiv A+B+C \mod p$, which will automatically prove $A\equiv C \mod p$. First of all note that $A+B+C \equiv \sum_{i=1}^{6m}(\frac{(-1)^i}{i}) \mod p$. As   
		\( \sum_{i=1}^{6m}(\frac{1}{i}) \equiv 0 \mod p \) (as each inverse is mapped bijectively to a non -  zero element, therefore it is just sum of all non zero elements, sum of whose is 0). 
		We get $\sum_{i=1}^{6m}(\frac{(-1)^i}{i}) \equiv 2\sum_{i=1}^{3m}(\frac{1}{2i}) \mod p$ which implies   
		 \[
		 A+B+C \equiv \sum_{i=1}^{3m}\frac{1}{i} \mod p
		 \]
		 Let $D= \sum_{i=1}^{m} \left( \frac{1}{2i-1} + \frac{1}{4m+2i}\right) $. We claim that $D \equiv 0 \mod p$. To prove that , first notice (using $1/a \equiv -1/(p-a) \mod p$ and  $\\ \sum_{i=a}^{i=b} f(i) = \sum_{i=a}^{i=b} f(a+b-i)$ respectively),
		 
		 $\sum_{i=1}^{m} \left( \frac{1}{4m+2i}\right) \equiv \sum_{i=1}^{m} \left( \frac{-1}{6m+1 - (4m+2i)}\right) \equiv \sum_{i=1}^{m} \left( \frac{-1}{6m+1 - (4m+2(m+1-i))}\right) \mod p $, this implies
		$ \sum_{i=1}^{m} \left( \frac{1}{4m+2i}\right) \equiv  \sum_{i=1}^{m} \left( \frac{-1}{2i-1}\right) \mod p $.
		Hence $D \equiv 0 \mod p$. 
		
		Now note that $3A \equiv \left(  \sum_{i=0}^{i=2m-1} \dfrac{3*(-1)^{3i+3} }{3i+3}\right) \equiv \sum_{i=1}^{i=2m} \dfrac{(-1)^i }{i}  \mod p$ .
		Now 
		$\sum_{i=1}^{i=2m} \dfrac{(-1)^i }{i}  \equiv 2D+\sum_{i=1}^{i=2m} \dfrac{(-1)^i }{i} \mod p $. This implies
		
		$\sum_{i=1}^{i=2m} \dfrac{(-1)^i }{i}  \equiv \sum_{i=1}^{m} \left( \frac{2}{2i-1} + \frac{1}{2m+i}\right)+\sum_{i=1}^{i=2m} \dfrac{(-1)^i }{i} \equiv \sum_{i=1}^{3m}\frac{1}{i} \mod p $, 
		
		Hence $3A \equiv A+B+C \mod p$ , as $A \equiv B \mod p$, we get $A \equiv C \mod p$. Since $A \equiv B \equiv C \mod p$, we have $P(x) \equiv A(1+x+x^2) \mod p$ ,hence $P(x) \equiv 0 \mod p$
		which proves our original proposition		
	\end{solution}
	
	\begin{exploration}
		Complex numbers!, altho not strictly needed gives a direction to the proof, since each cube root of unity is independent i knew we had to prove $A \equiv B \equiv C \mod p$. First part was trivial, tried and experimented many algebraic manipulation, looking into values of sum of inverses of 2 residue 3 using computer and many more things. Finally tried the value of the their sum since the sum looks somehwat pretty, it's clear you've to choose 0 residue to make this even somewhat solvable. Post that it was trivial . Main idea was to prove $A \equiv C \mod p$ indirectly using $3A \equiv A+B+C \mod p$ .Overall a medium-hard problem (for me) since main idea was clear
	\end{exploration}
	
	\begin{tags}
		Number Theory ,  roots of unity , John Berman (for searching using authors) , harmonic sums 
	\end{tags}
	
\end{document}