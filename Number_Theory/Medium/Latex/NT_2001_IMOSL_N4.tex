\documentclass{article}
\usepackage{amsmath}
\usepackage{amsthm}
\usepackage{amsfonts}
\usepackage{amssymb}
\usepackage{mathtools}
\title{IMOSL N4}
\date{\today}

\newtheorem*{problem}{Problem}
\newtheorem*{solution}{Solution}
\newtheorem*{lemma}{Lemma}
\newtheorem*{Proof}{Proof}
\newtheorem*{exploration}{Exploration}
\newtheorem*{tags}{Tags}

\begin{document}
	
	\maketitle
	
	\begin{problem}
		Let $p \geq 5$ be a prime number. Prove that
		there exists an integer $a$ with 
		$1 \leq a \leq p-2$ such that neither $a^{p-1} -  1$ nor  $(a + 1)^{p-1} -  1$ is
		divisible by $p^2$.
	\end{problem}
	
	\begin{solution}
		Note that if $p=5$, we choose $a=2$ , therefore we can assume $p \geq 7$
		\begin{lemma}
			For all $x$  we have $(x+p)^p \equiv x^p \mod p^2$
		\end{lemma}
		
		\begin{Proof}
			Note that $(x+p)^p \equiv \sum_{i=0}^{i=p} \binom{p}{i} p^i x^{p-i} \equiv x^p + \binom{p}{1} p x^ {p-1} \equiv x^p \mod p^2$
		\end{Proof}
		Note that the given condition is equivalent to finding a number $a$ such that neither 
		$a-a^{p} $ nor  $ (a+1) - (a + 1)^{p} $ is
		divisible by $p^2$. Let $f(a)=a-a^p$ then note that
		\\
		$f(a)+f(p-a) = p+ (a^p - (a-p)^p)$
		\\ 
		Therefore
		$f(a)+f(p-a) \equiv p \mod p^2$
		\\
		Hence at least one of $f(a)$ or $f(p-a)$ is not divisible by $p^2$.
		\\
		Let's suppose for the sake of contradiction that there exists a prime $p$ for which there exists no such $a$ s.t. both $f(a)$ and $f(a+1)$ are not divisible by  $p^2 $ for all $ 1\leq a \leq p-2$.Now if for some $t,1\leq t \leq p-2 $ both $f(t)$ and $f(t+1)$ are divisible by $p^2$ then this implies both of $f(p-t)$ and $f(p-t-1)$ are not divisible by $p^2$ which is a contradiction, hence only one of $f(t)$ and $f(t+1)$ are divisible by $p^2$. Hence the divisibility by $p^2$ alternates and as we have $f(1)=0$ , this implies $f(a) \equiv 0 \mod p^2$ for all odd $a$ s.t.  $1\leq a \leq p-1$. As 
		$f(a)+f(p-a) \equiv p \mod p^2$, we get $f(a) \equiv p \mod p^2$ for all even $a$ s.t.  $1\leq a \leq p-1$. Therefore we have 
		\\
		$f(2) \equiv p \mod p^2$ , hence $2^p \equiv 2-p $ and $f(3) \equiv 0 \mod p^2$ , hence $3^p \equiv 3 \mod p^2$ and $f(6) \equiv p \mod p^2$ , hence $6^p \equiv 6-p $. So we have 
		\\
		$2^p \cdot 3^p \equiv 6-p \mod p^2$ , hence
		\\
		$3(2-p) \equiv 6-p \mod p^2$ so $2p \equiv 0 \ mod p^2$ which is a contradiction.
		Hence our original claim was true. QED
		
	\end{solution}
	
	\begin{exploration}
		Lots of primitive root thinking, but simply adding in reverse works
	\end{exploration}
	
	\begin{tags}
		NT ,  Number Theory , modulo 
	\end{tags}
	
\end{document}
