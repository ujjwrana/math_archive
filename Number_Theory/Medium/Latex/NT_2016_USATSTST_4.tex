\documentclass{article}
\usepackage{amsmath}
\usepackage{amsthm}
\usepackage{amsfonts}
\usepackage{amssymb}
\usepackage{mathtools}
\title{USA TSTST/4}
\date{\today}

\newtheorem*{problem}{Problem}
\newtheorem*{solution}{Solution}
\newtheorem*{lemma}{Lemma}
\newtheorem*{exploration}{Exploration}
\newtheorem*{tags}{Tags}

\begin{document}
	
	\maketitle
	
	\begin{problem}
		Suppose that $n$ and $k$ are positive integers such that \[ 1 = \underbrace{\varphi( \varphi( \dots \varphi(}_{k\ \text{times}}n) \dots )). \] Prove that $n \le 3^k$.
		
		Here $\varphi(n)$ denotes Euler's totient function, i.e. $\varphi(n)$ denotes the number of elements of $\{1, \dots, n\}$ which are relatively prime to $n$. In particular, $\varphi(1) = 1$.
		
		Proposed by Linus Hamilton
	\end{problem}
	
	\begin{solution}
		\begin{lemma}
			\[
			\forall n \in \mathbb{N} , \frac{n}{\varphi(n)} \leq 3 \cdot \left( \frac{3}{2}\right)^{v_2(\varphi(n))-v_2(n)}
			\]
		\end{lemma}
		\begin{proof}
				Note that if $n=2^k$ for some $k\geq 0$ then this result holds trivially. Now let's assume that some odd prime divides $n$. Let $S$ denote the set of all primes dividing $n$ and $G$ denotes set of all odd primes dividing $n$.
				Then note that $v_2(\varphi(n)) \geq v_2(n) + |G|  -1$  as each odd prime will add at least one power of $2$ and only one power of 2 can be taken out from the even part.
				Hence $|G| \leq  v_2(\varphi(n))-v_2(n) +1 $
				Now note that. 
				\[
				 \frac{n}{\varphi(n)} = \prod_{p \in S}\left(1 + \frac{1}{p-1} \right) \leq 2\cdot \prod_{p \in G}\left(1 + \frac{1}{p-1} \right) \leq 2 \cdot \left(\frac{3}{2} \right)^{|G|} \leq 2 \cdot \left(\frac{3}{2} \right)^{v_2(\varphi(n))-v_2(n) +1 }
				 \]
				 which proves our original claim.
		\end{proof}
		Now let $f_t(m) = \frac{\varphi ^ {t-1} (m)}{\varphi ^ {t} (m)}$ where $t,m \in \mathbb{N}$ and $g^t(m) $ is the function $g$ applied to $m$ total $t$ times (if $t=0$ then $g^t(m)=m$). 
		Then note that 
		\[
		\prod_{i=1}^{i=k}  f_t(n) \leq \prod_{i=1}^{i=k} 3 \cdot   \left( \frac{3}{2}\right)^{v_2(\varphi^{t}(n))-v_2(\varphi^{t-1}(n))}
		\]
		Hence 
		\[
		n \leq 3^k \cdot   \left( \frac{3}{2}\right)^{-v_2(n)} \leq 3^k
		\]
		QED
	\end{solution}
	
	\begin{exploration}
		Basically the solution, looking into powers of 2 was the key. We can improve bound to $2 \cdot (3^{k-1})$ but this is clean
	\end{exploration}
	
	\begin{tags}
		NT , Euler totient , phi
	\end{tags}
	
\end{document}
